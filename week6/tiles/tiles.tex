\documentclass{article}
\usepackage[utf8]{inputenc}
\usepackage[english]{babel}

% Images
\usepackage{graphicx}

% Fonts
\usepackage{mathrsfs}
\usepackage{xcolor}

% Math libs
\usepackage{amssymb}
\usepackage{amsthm}
\usepackage{amsmath}

% Theorems
\newtheorem{theorem}{Theorem}
\newtheorem{corollary}{Corollary}[theorem]
\newtheorem{lemma}[theorem]{Lemma}
\newtheorem*{remark}{Remark}
\theoremstyle{definition}
\newtheorem{definition}{Definition}

% Page margin
\usepackage[margin=0.9in]{geometry}

%cpp
\usepackage{listings}
\usepackage{courier}

\lstset{
  basicstyle=\ttfamily\small,
  language=C++,
  columns=fullflexible
}

\title{Tiles}
\author{Klejdi Sevdari}
\date{\today}

\begin{document}

\maketitle

\subsection*{Problem restatement}
\noindent We model this problem as a graph. We consider each garden space to be a vertex, and we add an edge for each pair of garden spaces that are adjacent horizontally or vertically. With this formulation, the problem reduces to finding a maximum matching. 

\begin{figure}[!h]
    \centering
    \includegraphics[width=0.5\linewidth]{week6/tiles/tiles_graph.png}
    \caption{Graph reconstruction}
    \label{fig:graph}
\end{figure}

\noindent For this exercise, we need to go one step further and model it as a flow, so we transform the matching instance into a flow network. The standard reduction is the following: Every garden tile $v$ is split into two vertices: an \emph{in-vertex} $v_{\text{in}}$ and an \emph{out-vertex} $v_{\text{out}}$. In addition, we add two other vertices \emph{source} $s$ and \emph{sink} $t$. For each vertex $v$, we add edges $(s, v_\text{in})$ and $(v_{\text{out}}, t)$ each with capacity 1. Intuitively, this ensures that each tile can be used by at most one domino.\\

\noindent Next, for every pair of adjacent tiles $u$ and $v$ (sharing an edge in the grid), we add two directed edges
\[
u_{\text{in}} \to v_{\text{out}}, \qquad v_{\text{in}} \to u_{\text{out}},
\]
each with capacity~1. These edges encode the possibility of placing a domino that covers $u$ and~$v$.\\

\noindent A valid $s$--$t$ flow selects a set of disjoint tile pairs, each realised as a path
\[
s \to v_{\text{in}} \to u_{\text{out}} \to t,
\]
corresponding exactly to placing a domino on the pair $\{u,v\}$.

\begin{figure}[!h]
    \centering
    \includegraphics[width=0.55\linewidth]{week6/tiles/tiles_flow.png}
    \caption{Flow construction. Each edge has capacity 1.}
    \label{fig:placeholder}
\end{figure}

\subsection*{Proof of Correctness}
\noindent Observe that the maximum possible flow value is $|V|$, since each tile contributes at most one unit of flow entering at $v_{\text{in}}$ and one unit leaving at $v_{\text{out}}$. If a perfect matching exists, we can realise this upper bound directly: for every matched pair $(u,v)$, we route one unit of flow along both paths
\[
s \to u_{\text{in}} \to v_{\text{out}} \to t
\qquad\text{and}\qquad
s \to v_{\text{in}} \to u_{\text{out}} \to t,
\]
thereby saturating all source and sink edges. This constructs an $s$--$t$ flow of value $|V|$ whenever a perfect matching is present.\\


\noindent Conversely, assume that the garden does \emph{not} admit a perfect matching. Then there exists at least one tile $v$ that cannot be matched with any of its neighbours in any maximum matching. In the flow network, this means that every adjacency edge incident to either $v_{\text{in}}$ or $v_{\text{out}}$ must remain unused in any feasible $s$--$t$ flow, because using such an edge would correspond to matching $v$ with some neighbour. But the only way for flow to pass through $v$ is along a path
\[
s \to v_{\text{in}} \to u_{\text{out}} \to t
\]
for one of its neighbours $u$. Since no such path can be used, both the edge $(s,v_{\text{in}})$ and the edge $(v_{\text{out}},t)$ remain unsaturated. Thus at least one unit of the total available capacity from $s$ to $t$ is unused, and the value of any flow must satisfy $|f| < |V|$.\\

\noindent Combining both directions, we conclude that the flow has value $|V|$ if and only if a perfect matching exists in the original adjacency graph of tiles. Therefore, computing the maximum flow in this network correctly determines whether the garden can be completely tiled by dominoes.





\end{document}
